\documentclass{article}

\usepackage{polski}
\usepackage[utf8]{inputenc}
\usepackage{graphicx}
\usepackage{url}

\newcommand{\bb}{\textbf}


\title{Praca inżynierska}
\date{2017-10-01}
\author{Jędrzej Kozal, Karol Szpila}

\begin{document}

\begin{titlepage}
	\centering
	\includegraphics[width=0.25\textwidth]{logo_pol_wroclaw.png}\par\vspace{1cm}
	{\scshape\LARGE Politechnika Wrocławska \par}
	\vspace{1cm}
	{\scshape\Large Sieci neuronowe i neurosterowniki \par}
	\vspace{1.5cm}
	{\huge\bfseries Sprawozdanie z projektu \par}
	\vspace{2cm}
	{\Large\itshape Michał Leś, Jędrzej Kozal\par}
	\vfill
	prowadzący\par
	Dr ~Piotr \textsc{Ciskowski}

	\vfill

% Bottom of the page
	{\large 2017-10-01\par}
\end{titlepage}

\section{Wstęp}
Podstawowym założeniem projetku jest zaznajomienie się z neurosterownikami, ich zasadami działania oraz przygotowanie modelu predykcyjnego prostego obiektu dynamicznego. Dodatkowo przyjęto że bardziej interesujące będzie przyjęcie jakiegoś rzeczywistego obiektu niż badanie reakcji na teoretyczne charakterystyki jakie mogą posiadać obiekty.

Neurosterowniki są sposobem zaadresowania w Automatyce kwestii sterowania obiektami mocno nieliniowymi, z którymi tradycyjne metody sterowania jak sterowniki PID nie dają sobie rady. U podstaw działania neurosterowników leży idea działania sieci neuronowej, które ostatnio zdominowały pole uczenia maszynowego. Są one wykorzystywane w wielu dziedzianach nauki i techniki do rozpoznawania obrazów (computer vision), klasyfikacji, sterownia ruchem ulicznym, wspomagania użytkowników różnych aplikacji (jak np. podpowiadanie słów w trakcie pisania na smarphonie), czy nawet eksperymenty społeczne. W nimniejszej pracy postawiono zbadać w jaki sposób szerokie możliwości oferowane przez sieci neuronowe mogą być wykorzystane w automatyce.

\section{Podstawy teoretyczne}

\subsection{Przyjęty model obektu}

Obiekt w automatyce jest traktowany jako czarna skrzynka, która posiada wejścia i wyjścia. Na pobudzenie na wejściu reaguje odpowiedzią na wyjściu. Odpowiedź systemów dynamicznych zależy nie tylko od aktualnej wartości wejścia, ale także statnu obiektu w poprzednich chwilach. Zgodnie z tym można zapisać model nieliniowego obiektu dynamicznego jako:

\begin{equation}
\left\{	
\begin{array}{ll}
	\bb{x}(k+1) &= \phi (\bb{x}(k), \bb{u}(k)) \\
	\bb{y}(k)   &= \psi (\bb{x}(k))
\end{array} \right.
\label{rownanie_stanu}
\end{equation}

Układ \ref{rownanie_stanu} przedstawia równania stanu. Pierwsze równanie wiąże wewnętrzny stan obiektu z pobudzeniem, a drugie równanie wiąże stan obiektu z wyjściem. System opisywany tymi równaniami jest niezmienny w czasie (angl. time-invariant) - $\phi$ i $\psi$ nie zależą bezpośrednio od czasu. 

Jeśli zastąpimy funkcje nieliniowe przez odpowiednie macierze zyskamy system liniowy:

\begin{equation}
\left\{	
\begin{array}{ll}
	\bb{x}(k+1) &= A\bb{x}(k) + B\bb{u}(k) \\
	\bb{y}(k)   &= C\bb{x}(k) + D\bb{u}(k)
\end{array} \right.
\end{equation}

W rzeczywistych systemach w równaniach oprócz stanu $x$ i pobudzenia $u$ pojawiają się także zakłócenia $z$, które nie będą tutaj rozważane.

Klasa systemów liniowych jest od dawna dobrze znana. Znaleziono wiele sposobów badania i algorytmów sterowania obiektami liniowymi. O wiele trudniejsze w sterowaniu są obiekty nieliniowe. Nawet znając funkcję $\psi$ oraz $x(k)$ samo stworzenie modelu mogącego dokonać identyfikacji takiego obiektu jest trudne, ze względu na kumulujące się z czasem błędy numeryczne, oraz poprzez konieczność poczynienia założeń co do stabilności systemu.

Zadanie identyfikacji sprowadza się do wyznaczeniu modelu reagującego na poubudzenie w sposób zbliżony do reakcji rzeczywistego obietku:

\begin{equation}
	|| \hat{y} - y || < \epsilon
\end{equation}

\section{Zrealizowane zadania oraz otrzymane wyniki}


\section{Wnioski}



\newpage
\begin{thebibliography}{9}

\bibitem{Osowski}
Stanisław Osowski.
\textit{Sieci Neuronowe w ujęciu algorytmicznym}
Wydawnictwo Naukowo-Techniczne, Warszawa 1996

\bibitem{wav} 
Strona prowadzącego,
\\\texttt{http://staff.iiar.pwr.wroc.pl/piotr.ciskowski/skrypt/SNwMATLABie.htm}

\end{thebibliography}


\end{document}