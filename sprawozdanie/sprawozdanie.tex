\documentclass{article}

\usepackage{polski}
\usepackage[utf8]{inputenc}
\usepackage{graphicx}
\usepackage{url}


\title{Praca inżynierska}
\date{2017-10-01}
\author{Jędrzej Kozal, Karol Szpila}

\begin{document}

\begin{titlepage}
	\centering
	\includegraphics[width=0.25\textwidth]{logo_pol_wroclaw.png}\par\vspace{1cm}
	{\scshape\LARGE Politechnika Wrocławska \par}
	\vspace{1cm}
	{\scshape\Large Sieci neuronowe i neurosterowniki \par}
	\vspace{1.5cm}
	{\huge\bfseries Sprawozdanie z projektu \par}
	\vspace{2cm}
	{\Large\itshape Michał Leś, Jędrzej Kozal\par}
	\vfill
	prowadzący\par
	Dr hab.~Piotr \textsc{Ciskowski}, prof. nadzw. PWr

	\vfill

% Bottom of the page
	{\large 2017-10-01\par}
\end{titlepage}

\section{Wstęp}
Podstawowym założeniem projetku jest zaznajomienie się z neurosterownikami, ich zasadami działania oraz przygotowanie modelu predykcyjnego prostego obiektu dynamicznego. Dodatkowo przyjęto że bardziej interesujące będzie przyjęcie jakiegoś rzeczywistego obiektu niż badanie reakcji na teoretyczne charakterystyki jakie mogą posiadać obiekty.

Neurosterowniki są sposobem zaadresowania w Automatyce kwestii sterowania obiektami mocno nieliniowymi, z którymi tradycyjne metody sterowania jak sterowniki PID nie dają sobie rady. U podstaw działania neurosterowników leży idea działania sieci neuronowej, które ostatnio zdominowały pole uczenia maszynowego. Są one wykorzystywane w wielu dziedzianach nauki i techniki do rozpoznawania obrazów (computer vision), klasyfikacji, sterownia ruchem ulicznym, wspomagania użytkowników różnych aplikacji (jak np. podpowiadanie słów w trakcie pisania na smarphonie), czy nawet eksperymenty społeczne. W nimniejszej pracy postawiono zbadać w jaki sposób szerokie możliwości oferowane przez sieci neuronowe mogą być wykorzystane w automatyce.

\section{Podstawy teoretyczne}


\section{Zrealizowane zadania oraz otrzymane wyniki}


\section{Wnioski}



\newpage
\begin{thebibliography}{9}

\bibitem{Osowski}
Stanisław Osowski.
\textit{Sieci Neuronowe w ujęciu algorytmicznym}
Wydawnictwo Naukowo-Techniczne, Warszawa 1996

\bibitem{wav} 
Specyfikacja nagłówka pliku wav,
\\\texttt{http://soundfile.sapp.org/doc/WaveFormat}

\end{thebibliography}


\end{document}